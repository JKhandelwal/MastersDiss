\section{Scope}
\label{scope}
GDELT is a large resource which accumulates vast quantities of data each day, thus by necessity, the scope of the dissertation has to be restricted. There were several initial approaches, firstly looking at one day's data on GDELT, and one article to test out and try the topic modelling approaches. The headlines were used from the source URLs instead of the entire articles. This is due to the fact that it would be infeasible to retrieve each article from the source URL and process in an automated fashion, as each article and site would have a completely different HTML structure, and automating the parsing would be beyond the scope of this dissertation. Furthermore, some articles may be beyond paywalls, which would be difficult to reach with conventional web scraping methods.

The final data which was tested was the time period between the 1st of March 2020 and the 30th of April 2020 (inclusive). To try to curate the topic model, the GDELT filtering system was used, to only look at USA/China related media. 

The stock market which was used was the Dow Jones Index. This was used as it was thought that this would give the best potential correlation between the GDELT data and stock change. 

It should be noted that GDELT is a database which is updated on a daily basis with any missing backdated information being added. Thus the March and April data used in this dissertation was collected on the 11th of June 2020, and the May June data used was collated on the 24th of July 2020. 

\subsection{Code and Language}
The results and code for this dissertation were all written in the python programming language. This was chosen as there were many existing libraries for the algorithms used in this dissertation, along with significant support for building classification models. 

