\documentclass[a4paper]{article}

%% Language and font encodings
\usepackage[english]{babel}
\usepackage[utf8x]{inputenc}
\usepackage[T1]{fontenc}

%% Sets page size and margins
\usepackage[a4paper,top=3cm,bottom=2cm,left=3cm,right=3cm,marginparwidth=1.75cm]{geometry}

%% Useful packages
\usepackage{filecontents}
\usepackage{amsmath}
\usepackage{algorithm}
\usepackage[title]{appendix}
\usepackage{algorithmic}
\usepackage{listings}
\usepackage{graphicx}
\graphicspath{{images/}}
\usepackage{float}
\usepackage{listings}
\usepackage{amssymb}
\usepackage{latexsym}
\usepackage{amsfonts}
\usepackage{epsfig}
\usepackage{mathrsfs}
\usepackage{mathtools}
\usepackage{enumitem}
\usepackage{bashful}
\usepackage{array}
\usepackage{subfig}
\usepackage{url}
\usepackage[colorinlistoftodos]{todonotes}
\usepackage[colorlinks=true, allcolors=blue]{hyperref}
\usepackage{titlesec}
\usepackage{enumitem}
\usepackage[title]{appendix}
\setlength{\parindent}{0pt}
\setlength{\parskip}{1.5em}

\setcounter{secnumdepth}{4}

\titleformat{\paragraph}
{\normalfont\normalsize\bfseries}{\theparagraph}{1em}{}
\titlespacing*{\paragraph}
{0pt}{3.25ex plus 1ex minus .2ex}{1.5ex plus .2ex}
\begin{document}
\title{	Using NLP and Predictive Modelling on the Stock Market with the Global Database for Events Language and Tone


Using Geopolitical Events in the Global Database for Events Language and Tone with NLP and Predictive Modelling on Stock Market Data

Using NLP Geopolitical events in the Global Database for Events Language and Tone and Predictive Modelling 

Using Global Database of Events, Language, and Tone with to Cluster Geopolitical Events 

}
\author{}
\date{}
%\setcounter{secnumdepth}{-2}
 \thispagestyle{empty}
	
	\begin{titlepage}
		\vspace{\fill}
		\maketitle
		 \thispagestyle{empty}
		\begin{figure}[H]
			\centering
			\includegraphics[width=0.4\textwidth]{Images/SchoolLogo.png}
		\end{figure}
		\begin{center}
			\Large\textbf{University of St Andrews}\\
			\vspace{1cm}
			\Large\textbf{School of Mathematics and Statistics}
		\end{center}
		
		\vspace{\fill}
		\Large%\textbf{Details}
		\begin{itemize}
			\renewcommand\labelitemi{--}
			\item[] \textbf{Supervisor}: Dr Carl Donovan  
			\item[] \textbf{Module}: MT5099 Masters Dissertation in Partial Fulfillment of the MSc.\ in Data Intensive Analysis
			\item[] \textbf{Date}: \today
		\end{itemize}
		\vspace{\fill}
	\end{titlepage}



%\setcounter{secnumdepth}{-1}
\section*{Declaration}
\addcontentsline{toc}{section}{Declaration}
I hereby certify that this dissertation, which is approximately 14000 words in length, has been composed by me, that it is the record of work carried out by me and that it has not been submitted in any previous application for a degree. This project was conducted by me at University of St Andrews from June 2020 to August 2020 towards fulfilment of the requirements of the University of St Andrews for the degree of MSc. in Data Intensive Analysis under the supervision of Dr Carl Donovan.

17th August 2020

Jalaj Khandelwal
\section*{Abstract}
\addcontentsline{toc}{section}{Abstract}
This Masters Dissertation was focused on trying to use geopolitical events to predict changes in the stock market. This aim was to first use the Global Database for Language and Tone (GDELT) to build topic models, which organise the geopolitical news into topics which could be used as a filter for the daily news cycle. Then, using the Tone and Geopolitical conflict scores provided by GDELT, models would be made made to predict changes in the stock market. There were two main topic modelling approaches, firstly using Latent Dirichlet Allocation, and secondly using K-Means clustering with TF-IDF. Both of these approaches either were incapable, or struggled to filter unseen news effectively, which means that they would not be effective in picking up new topics. However, many different stock predictive models were tried, and it was found that the accuracy of models using the GDELT information was higher in predicting the direction the stock market would go, compared to a model which just used previous stock direction values. The best of the models also were able to demonstrate adding value to a simple portfolio, which suggests that using GDELT is a viable metric by which to predict the stock market. 
\section*{Impact of COVID-19}
The impacts of COVID-19 was minimal on this dissertation. The only issue was the amount of data used being restricted due to the machine constraints. Without Covid-19 data covering a longer time period from GDELT would have been used.
\newpage
\tableofcontents
\listoftables
\newpage
\listoffigures


\section{Introduction}
Using geopolitical news to predict stock prices would be of great interest to investors. If, by mining geopolitical news predictions are possible and accurate with regards to the direction the stock market will go, it would mean investors will be able to not only not \textit{lose} money, e.g. by selling in a bear market before the price crashes, but in certain circumstances, be able to make money, e.g. buying in a bull market before the price increases.

This dissertation was focused around using the Global Database of Events, Language, and Tone (GDELT) to build topic models which could be used to predict the changes in the stock market. The topic models and the GDELT data used was related to geopolitical information. The aim was to use GDELT data to build a topic model which could be used to filter news information for specifically geopolitical news. Then the geopolitical measure of the conflict cooperation Goldstein Scale for geopolitical events provided by GDELT would be used to try and predict information on the stock market.

The methodology behind building a topic model took news headlines from the GDELT data over a period of time, and then performed cluster analysis/topic curation on the headlines themselves, with the aim being to tease out the underlying topics, which would ideally be specifically focused on the geopolitical topics. The next step was to use the curated topic model to filter other news to end up with headlines specifically related to that geopolitical topic. Then a variety of models could be tested to aim to predict the stock market shifts.

The initial approach to building the topic models was to use existing topic modelling algorithms. The first algorithm used was the Latent Dirichlet Analysis on data from the GDELT Events Table, filtered to data which which was specific to events concerning the USA or China. The aim for this data would be to sort all of the words present into M topics. USA and China specific data was chosen for two reasons. Firstly it was to narrow down the dataset, as GDELT is a very large resource, and it would be simpler initially to run on a simpler dataset. The second reason for this is using geopolitical news would have to be targetted considerably to specific events in specific countries and specific markets. Over the time period from March to April, there were significant events which took place specifically between those two countries, and it was thought this would provide the best opportunity to find signal within the data. 

However, this approach would not allow filtering by calculating the distance between the topics and unseen news, thus a different model was used. The K Means clustering algorithm was used alongside the Term Frequency Inverse Document Frequency (TF-IDF) to build clusters from the words in the headlines. This would mean that the clusters would be built based on words and phrases which occur frequently. It was hoped those clusters would represent the words and the phrases which are relevant to geopolitical information, and thus it could be used. This approach also did not work as initially thought, as due to the TF-IDF vectorisation the algorithm was unable to perform filtering in a usable fashion as it did not filter relevant geopolitical news effectively from noise. Thus this was not used to filter news for the modelling approaches, instead the modelling was on the entirety of the selected USA China data, without any further filtering. 

The modelling worked on the basis of predicting shifts in the Dow Jones Industrial Average. This was chosen again because geopolitical risk prediction would be done on specific markets, and it was thought that any events concerning the USA and China would be reflected into the Dow Jones. When making predictive models, stock shift was used for predictions as opposed to predicting exact stock price. It would also be assumed that the effect of events would not be singular, and instead effect the market for several days. Thus several different models with different styles of lagged Goldstein Scale and Average Tone values were modelled against the stock shift. The best model picked was a Random Forests Classifier which was used to predict in the interval of May and June 2020, compared against a model which just used the information from the previous day to predict whether the stock price increased or decreased. 

This model accrued a higher accuracy of predicting the stock shift over the prediction interval, however there is a large scope for further modelling to explore both GDELT and topic modelling further, both in terms of the scale of the data being predicted, and different markets and modelling strategies. 

This report will cover relevant background literature in Sections 2 to 4, an overview of the experiments performed and results in chapters 6 and 7, and a discussion of the results in chapter 8.
\section{Existing Work}
\label{prior}
There exist several approaches to model and predict stock prices. These approaches use many different sources of data and varying modelling styles. However, there remains limited prior work in modelling stock prices using calculated Geopolitical risk, though there are limited approaches to calculating geopolitical risk on its own.

\subsection{Geopolitical risk}
The majority of approaches in modelling Geopolitical risk work on using news information along with manually defined terms of geopolitical risk. One of the most recent approaches was done by Dario Caldara and Matteo Iacoviello, where they created a Geopolitical risk index from a series of news articles \cite{caldara2018measuring}. 

This risk index was calculated by manually creating search terms which define geopolitical risk in several categories, and then over time calculate the percentage of articles of selected number of news media in which the search terms appear. This is then used to model the stock prices.

There are other risk measures present, Black Rock also have a publicly viewable index called the BlackRock Geopolitical Risk Indicator (BGRI) \footnote{https://www.blackrock.com/corporate/insights/blackrock-investment-institute/interactive-charts/geopolitical-risk-dashboard}, which works in a similar fashion where key words are selected which define geopolitical risk, followed by an application of a sentiment score from a list of predefined words, and then those two scores are combined to make the BGRI total score. Furthermore, there are also many other organisations which almost certainly will model Geopolitical Indices, such as Reuters, but will not disclose the exact modelling for proprietary purposes.

One of the major differences between the existing approaches in modelling geopolitical risk and the approach used in this dissertation, is here we incorporate the cooperation conflict scale provided in GDELT. GDELT also differs from the data sources used in existing modelling approaches. In existing approaches, only manually selected news media are used for the data, whereas GDELT does not restrict which news are stored and calculated, and thus has a much wider scope in terms of news and events stored. Furthermore, this approach attempts to use topic modelling to filter the news, as opposed to using manually defined search terms, in an attempt to make the prediction process more automated. 

\subsection{Modelling Stock Prices}
There is a large vested interest for people to model the stock market effectively, as any investor with extra knowledge about where the market will go will be able to take a position most beneficial to them. Thus, since having accurate predictive models are so beneficial, there exist many models which aim to predict market changes.

The Efficient Markets Hypothesis asserts that current securities prices reflect all available information and expectations\cite{fama1960efficient}. This is related to random walk theory, which in finance literature is used to describe a price series where all price changes are random departures from earlier prices, which means that any specific day's price change is not related to the previous day's price and only on the news that is received on that day. Whilst this theory has become more controversial \cite{malkiel2003efficient}, it is accepted that stock prices are related to new news. This means that Geopolitical events which generate news, and by extension an index of risk which track them, should have a measurable effect on stock prices. For example, the risk would increase if a `bad` event occurs, such as an increase in international conflict, or a negative piece of news is published. Since the risk would be higher it could be assumed that stock prices would decrease as investors could be less likely to buy and more likely to sell which would cause the decrease in the price. If the reverse were to happen with the risk index, it could be assumed the opposite would happen in the market.

As stock market modelling has existed for many years there have been several esoteric factors used to model and predict stock prices aside from the news, such as the weather on Wall Street \cite{saunders1993stock}. For most of the non quantitative data, sentiment analysis is commonly used to help predict stock prices, where natural language processing is used to gain a measure of tone in the data being used. The tone is a quantitative measure of how positive or negative the text in the data being used. This is then used to model the price changes. A number of studies shows that large amounts of data on social medias such as Facebook and Twitter can be useful for forecasting economic indicators such as the stock market \cite{bollen2011twitter} \cite{arias2014forecasting}. 

There are many different machine learning and statistical models used for modelling stock prices. One of the simplest approaches to stock price modelling is to perform autoregressive modelling on purely the time series data for stock prices. In this approach, the only thing which is used to predict the prices are the past values of the stock. One common approach is to use an autoregressive integrated moving average model (ARIMA) \cite{ariyo2014stock}. This is part of a set of models where the response variable, Y, is an ordered time series, where the Y for any given time is dependent on and calculated only from previous values in the time series. 

When predicting stock prices from variables other than the previous prices, many different machine learning algorithms are used. Aside from classical logistic regression models, these include SVMs \cite{cao2003support}, Random Forests \cite{khaidem2016predicting}, and Neural Networks \cite{egeli2003stock}, along with Bayesian approaches such as Naive Bayes \cite{khedr2017predicting}. These models can be used to predict both the stock prices, with a numeric response variable, and binary stock shift \cite{nguyen2015sentiment}, with a class response.  

There is another concern with modelling on the stock market. When predicting any stock market, one of the major issues which arises is that in the long term the stock market always grows, for the S and P for example there is an annualised growth of on average 9.5\% \cite{sp} without the affect of inflation factored in. Thus any investor investing over a sufficiently long period of time will nearly always make money. The aim then, is to build models which would give better results than the normal stock market rise, and thus any strategy or predictions by any model would have to prove and provide better returns than just investing in the long term growth of the stock market.

A mixture of these algorithms were used alongside the GDELT algorithm to try and predict stock shift. 


\section{GDELT}

The Global Database for Events Language and Tone was the basis for this dissertation. This is a database which stores a record of every broadcast, print, and web news published since 1979, in over 100 languages from all parts of the world. It contains over a quarter of a billion georeferenced records. 

The records are stored across multiple tables accessible either directly, or through Google Big Query. The main table which was the source of the data in this dissertation was the Events table. This table maintains the record of all events, along with cursory information about each event, such as which countries it happened in or between, the dates, and the URLs of the first news article this appeared in.

The interesting part of the events table is the source URL, the Average Tone and the Goldstein Scale. This dissertation and prototyping works on the basis of the headline if it is present in the source URL. The reason for this is that it would be infeasible to retrieve each article and process the documentation, as each article and site would have a completely different html structure, and automating the parsing would be almost impossible. Furthermore, some articles may be beyond paywalls, which would be difficult to reach with conventional web scraping methods such as wget.

For each event, there are two columns of note, the Average Tone and the Goldstein Scale. The Average Tone is a measure of the sentiment of all the news media which describes the specific event. It is on a scale of -100 (Extremely Negative) to +100 (Extremely Positive). The most common values observed are usually between -10 and 10, and 0 indicated a neutral tone. 

The Goldstein Scale is the main metric by which geopolitical modelling occurs, and was initially defined as the Conflict-Cooperation Scale for Work Environment Impact Scale (WEIS) events data \cite{goldstein1992conflict}. This is a scale between -10 and +10 which captures the potential impact an event will have on the stability of the country, with -10 being the worst impact an event can have and +10 being the most positive impact an event can have. It should be noted that these numbers are predefined and do not take into consideration the extent to which an even occurs, for example event type 071, Extending economic aid; give, buy, sell, or borrow, is given a score of +7.4, regardless of whether it is £1 of aid given or £1 million. 

GDELT has been used previously for stock predictions in the past \cite{memari2017predicting} \cite{alamro2019predicting}, however, it is used much more commonly used for predicting and analysing geopolitical events themselves \cite{qiao2017predicting} \cite{yonamine2013predicting}. 
\section{Topic Modelling}
\label{topic}
Topic Modelling is an unsupervised machine learning and statistical modelling approach which aims to discover the abstract `topics` in a collection of documents, referred to as the corpus. This can be used to find hidden semantic structures in a document. This works by looking at how often words appear in the corpus and trying to create topics made of words which are related to each other. For example an article about dogs would be more likely to have words such as `dog` and `bone`, and thus the topic model would most likely group those two words together into a topic. 

There are many approaches to topic modelling, one of the foremost examples is Latent Dirichlet Allocation \cite{blei2003latent}, which aims to reverse-engineer the topics by assuming the text was created from K topics and all words directly relate to one of the K topics. There are other forms of clustering algorithm which can be applied to words, such as the K Means clustering algorithm. 

\subsection{Latent Dirichlet Allocation}
Latent Dirichlet Allocation (LDA) is an unsupervised learning model which aims to collate words into K topics. When performing LDA, the text to be analysed is split into a series of documents, each composed of a bag of words where the order of the words does not matter. Most of the time, this will require pre-processing to remove words which do not contribute to the topics present in the work such as `the`, `is`,  and `a`. 

\noindent This algorithm assumes that the documents was made by first picking K topics, and any words present in the document belong to one of those K topics. The algorithm aims to reverse-engineer this process.
% START HERE 

Initially if the corpus is composed of a set of documents, D, the algorithm would first perform tokenisation so that each individual word is treated as a unique identity. Then each of the tokens for each of the documents would be parsed to remove the stop words. At this point each document is a list of tokens representing words which are not stop words. From this a document term matrix is created. This matrix has the dimensions $m$ x $v$, where M is the number of documents present in the corpus. V is the number of unique words present across all of the documents, or the vocabulary. This document term matrix records the term frequency of all of the words in the vocabulary for all of the documents present.

\begin{figure}[H]
	\centering
	\includegraphics[width=0.6\textwidth]{images/LDA.png}
	\caption{A Graphical Plate representation of (smoothed) LDA (image from \cite{LDA})}
	\label{fig:ldafigure}
\end{figure}

The reverse-engineering process is shown graphically in Figure \ref{fig:ldafigure}. M denotes the number of documents, N refers to an individual document, and W refers to a single word, and is the only observable variable in the system. The algorithm assumes several matrices. \boldmath{$\varphi$} is defined as the word distribution across topics. In practice this is a matrix where \boldmath{$\varphi_{k}$} is the probability distribution across the V for topic K, such that \boldmath{$\varphi_{j, k}$} represents the probability that the \unboldmath{$j^{th}$} word in the vocabulary belongs in topic K. \boldmath{$\theta$} is defined as the topic distribution across documents, which means \boldmath{$\theta_{i}$} represents the topic distribution for document i, and that \boldmath{$\theta_{i, k}$} represents the probability of topic k being in document i. \textbf{}{Z} represents the matrix of documents and topics, where \textbf{$Z_{i,j}$} is the topic for the \unboldmath{$j^{th}$} word in document i. 

$\alpha$ and $\beta$ are the external parameters which control the initial distributions. $\alpha$ is the parameter which initially sets the shape of the topic distribution across documents, \boldmath{$\theta$}, and \unboldmath{$\beta$} is the parameter which initially sets the word distribution across topics, \boldmath{$\varphi$}. The aim is to optimise parameters \unboldmath{$\alpha$} and $\beta$ to find the best distribution of word and document probabilities which have generated the corpus most accurately. In the original paper \cite{blei2003latent}, both the topic-word distribution, $\beta$, and the topic-distributions, $\alpha$ can be modelled using a sparse Dirichlet prior, as it would be thought that the probability distribution of words in a topic and documents across topic would not necessarily be symmetric, and not all documents/words would contain all topics. Large values in $\alpha$ push the document topic distribution towards being more balanced between topics, and smaller alpha values push the document topic distribution probabilities towards being weighted more towards certain topics than being weighted evenly. 

The posterior probability is represented as below:
\begin{align*}
	p(\varphi_{1:k}, \theta_{1:M}, z_{1:M} | D; \alpha_{1:M}, \beta_{1:K})
\end{align*}

This can be calculated using variational inference, as the probability defined above is intractable. This entails calculating an approximation of the true posterior probability, and minimising the difference between the true posterior and the estimated posterior. In this case, the difference between the true and approximated posterior is the distance between them. To obtain the most accurate approximation, this distance has to be minimised which in this case is achieved by minimising the KL divergence between the approximation and the true posterior probability. The optimisation is shown below:
\begin{align*}
	\gamma^{*}, \phi^{*}, \lambda^{*} = argmin_{\gamma^{*}, \phi^{*}, \lambda^{*}} D(q(\varphi, \theta, z, | \gamma, \phi, \lambda) ||p(\varphi, \theta, z | D; \alpha, \beta))
\end{align*}


$\gamma$, $\phi$, and $\lambda$ are the free variational parameters used to approximate $\theta$, z, and $\varphi$ with. D(q||p) represents the KL divergence between q and p. Changing the $\gamma$, $\phi$, and $\lambda$ parameters changes the distance between the estimate, q, and the true posterior, p, and the aim is to find the values which minimises that distance.

In algorithmic terms this works on the basis of optimising one of $\varphi$, $\beta$, and \textbf{z} at a time. This is because these matrices are are intrinsically linked to one another. Pseudocode for the algorithm is shown below (Algorithm taken from \cite{ldaalgorithm}):
\begin{algorithm}
	\begin{algorithmic}
	\STATE Initialise Topics based on $\alpha$ and $\beta$\\
	repeat\\ 
	\hspace{1cm} for each document do\\
			\hspace{2cm} repeat\\ 
				\hspace{3cm}Update the topic assignment Variational parameters ($\theta$)\\
				\hspace{3cm}Update the topic proportions Variational parameters ($\varphi$)\\
			\hspace{2cm}until document objective converges\\
		\hspace{1cm}end for\\ 
		\hspace{1cm}update topics from aggregated per-document parameters (\textbf{z})\\
	until corpus objective converged\\
		\end{algorithmic}
\end{algorithm}

LDA has been used widely for topic modelling across many fields.  This has also been used in stock market predictions by attempting to mine many different kinds of data to used as prediction. This includes seeing if anything from social media data \cite{nguyen2015sentiment}, to topics in financial news \cite{feuerriegel2016analysis} affect stock prices. This is similar to the goal of this dissertation, and thus this was deemed a suitable metric to use for this purpose. 
 % END HERE
 \subsection{Term Frequency-Inverse Document Frequency}
 TF-IDF is a well known numeric statistic used to calculate the importance of a word to a document in a collection or corpus. It works based on of creating a weighting for each word, based on the product of the term frequency and the inverse document frequency. The term frequency is the count of how many times each word appeared in the document. The inverse document frequency aims to measure how much information a word provides to the document, i.e. if the word appears extremely often (e.g. the word `the`), it would attain a lower IDF score, and vice versa. It is calculated by taking the logarithm of the inverse of the fraction of documents which contain that word. The final TF-IDF value is calculated by multiplying the term frequency and inverse document frequencies together This is shown below:
 
 \begin{center}
 	$TF(t,d) = f_{t,d}$\\
 	$IDF(t, D) = log\frac{N}{\{d \in D : t \in D\}}$
 	$TF-IDF(t, d, D) = TF(t, d) * IDF(t, D) $
 \end{center}
 
The term frequency, \textit{tf}, for a term \textit{t}, in a document \textit{d}, is found by calculating the frequency of term \textit{t} in document \textit{d}. The inverse document frequency, for term \textit{t}, in a set of documents, \textit{D}, is the log of the number of documents in the corpus, \textit{N},  divided by the number of documents, \textit{d}, in the set of documents, where term \textit{t} is in the document.  
 
 TF-IDF is used extensively in text mining, as it shows the most important words of a corpus, this is used extensively, from paper recommender systems \cite{beel2016paper}, search engines \cite{xu2014pos}, and digital libraries \cite{philip2014application} alongside other uses. This has also been used extensively for predicting stock prices, as part of a wider prediction using sentiment analysis model. 
 
 
\subsection{K Means Clustering}
Another methodology of grouping objects together is to use a clustering technique such as the K Means algorithm. This algorithm, given a matrix X, of dimension \textit{n} x \textit{p}, where each row vector represents a point in p-dimensional space, places K candidate cluster centres randomly in p-dimensional space. Each of the points in p-dimensional space is allocated to the closest cluster. This is found for each point by calculating which cluster centre has the minimal distance to that point. 

The distance metric used for calculating distances between points in p-dimensional space can vary but it is most often the Euclidean Distance. The algorithm is an optimisation problem which aims to minimise the Within Cluster Sum of Squares, which is also the cluster variance. After assigning the points to the clusters, the location of the cluster centres are shifted to the mean of all of the points assigned to that cluster. Then the distances to the cluster centres are recalculated for all of the points, and the points are reassigned clusters to the cluster who's centre is the closest. This process of moving the cluster centres and reassigning the points continues until the assignments of the points to the clusters do not change. It should be noted that this algorithm is heavily dependant on the starting positions of the cluster centres, and is not guaranteed to find the optimal solution. As such it remains a computationally NP hard problem \cite{vattani2009hardness}. 
 
The K Means clustering algorithm only works with numerical data in n dimensions as it needs to quantify the distance between points. Thus if K means were used to cluster words, the words would have to be transformed into a numeric representation. A naive solution would be to just use the ASCII values of the words. However to perform clustering effectively based on the meaning of the words and relevance to the text as a whole, the numerical value would preserve any underlying relationship between the words and the document overall, which is not possible if ASCII was the one which was used. TF-IDF would be such a method, as it weights the importance of a word to a text, and the frequency of usage. 

This combined method of using TF-IDF with K Means is widely used. It has been used for summarisation of document spaces \cite{khan2019extractive}, for classification \cite{buana2012combination}, and specifically for topic detection \cite{6066301}. Thus, this was considered a suitable approach for topic classification.

\subsection{Mahalanobis distance}

Using the Euclidean distance for word clusters often presents a challenge, as the clusters may not end up being spherical in nature \cite{raykov2016k}, thus a different metric can be used, the Mahalanobis distance. This can also be used to calculate the distance between 2 points. This is calculated by measuring the number of standard deviations between points a and b, and can generalise to higher dimensions via the variance/covariance matrix.

The Mahalanobis distance would mainly be used after the cluster has been fitted (since the final variance covariance matrix is required), to calculate the distance between a point and the centres of the clusters. The Mahalanobis distance calculation is shown below. \textbf{X} is a matrix of $n$ vectors, such that \boldmath$x_{i}$ is a vector which represents a point in p-dimensional space. \boldmath$X_{c}$ represents the matrix where the points have been centred, and thus the variance covariance matrix can be calculated by matrix multiplying the transpose of the centred matrix and the centred matrix and dividing by the n -1, where n is the number of points. $\bar{ x  }$ refers to a vector, in this case a centroid, which represents a point in p-dimensional space. Thus the Mahalanobis distance is the square root of the difference between the point $x_{i}$ and the cluster centre$\bar{ x  }$, multiplied by the inverse of the variance covariance matrix, multiplied by the transpose of the difference between $x_{i}$ and the cluster centre. 


\begin{center}
	Covariance Matrix:
	\boldmath$C_{x} = \frac{1}{n - 1} (X_{c}) ^T (X_{c})$
	
	Mahalanobis Distance:
	\unboldmath$MD_{i}$ = \boldmath$\sqrt{(x_{i}  -  \bar{ x  } ) C_{x}^{-1} ( x_{i}   - \bar{ x  }  )^T    }  $
\end{center}

The Mahalanobis distance has been used with clustering algorithms such as k-means \cite{melnykov2014k} \cite{cerioli2005k}, and is frequently used in distributions of clusters which are either elliptical \cite{mitchell1985mahalanobis} or non normal distributions \cite{warren2011use}. Since the distribution of points around the k means clusters is likely to be unknown if TF-IDF is used to numerically transform the words, this was seen as a suitable metric to use for evaluating the clusters the algorithm would create.
\section{Scope}

GDELT is a large resource which accumulates vast quantities of data each day, thus by necessity, the scope of the dissertation has to be restricted. There were several initial approaches, firstly looking at one day's data on GDELT, and one article to test out and try the topic modelling approaches. 

The final data which was tested was the time period between the 1st of March 2020 and the 30th of April 2020 (inclusive). To try to curate the topic model, the GDELT filtering system was used, to only look at USA/China related media. 

The stock market which was used was the Dow Jones Index, which measures the stock performance of 30 companies on US stock markets. This was used as it was thought that this would give the best potential correlation between the GDELT data and stock change. 

\subsection{Code and Language}
The results and code for this dissertation were all written in the python programming language. This was chosen as there were many existing libraries for the algorithms used in this dissertation, along with significant support for building classification models.

\subsection{Impact of COVID19}
The impacts of Covid\-19 was minimal on this dissertation. The only issue was the that I was restricted on the amount of data used. I would have used more data time wise on GDELT to test my data on.
\section{Experiments}

\subsection{Topic Modelling}


\subsection{Stock Modelling}

The main aim for this type of modelling was to predict whether the stock shifted up or down. Data was taken from the Dow Jones Industrial average which measures the stock performance of 30 large companies on stock exchanges across the United States, namely the NASDAQ and the New York Stock Exchange. Each day's difference was calculated between the opening and closing prices, and either a 1 or a -1 was decided to represent the stock market going up and down. For the scope of the project, and in line with other predictive modelling approaches, it was decided to only predict either the market going up or down. \\


\subsection{Preprocessing}
There was substantial preprocessing required for the data, first of all the stock market does not open on weekends or other holidays, however the news cycle very much does, thus the news over weekends was collated and averaged into the Friday figures. This meant of course that the prediction data had to the shifted, to ensure that information from the future was not being used to predict the data.\\

The next issue to consider whilst preprocessing the data was the issue of lag modelling. It is reasonable to expect that if there is an underlying relationship between the Goldstein Score/Average Tone and the daily stock price day, it isn't restricted to just the previous day's news, but instead could be a few days worth of modelling. Thus the average scores and the Goldstein scales would have to be smoothed using several moving window calculations.




\section{Results}
\subsection{Dow Jones and GDELT}
The data from the Dow Jones and the GDELT Average Tone and Goldstein Scale is presented. Figure \ref{fig:dow_diff} shows the Dow difference between open and close prices on a daily basis. This data is fairly chaotic, with the difference jumping between positive and negative frequently and without an apparent pattern, thus a moving average of 3 days was plotted, which also remains slightly chaotic, but more trends appear. The market difference appears to be getting negatively larger towards the middle of the time period before recovering towards 0. 

\begin{figure}[H]
	\centering
	\subfloat[Dow Jones Daily Difference]{  \includegraphics[width=0.45\textwidth]{images/dow_diff.png}\label{fig:diff}}
	\subfloat[Dow Jone Moving Average]{  \includegraphics[width=0.45\textwidth]{images/dow_diff_ma.png}\label{fig:dow_diff_ma}}\\
	\caption{Dow Jones Daily Difference and Daily Difference Moving Average}
	\label{fig:dow_diff}
\end{figure}

Figure \ref{fig:avg_tone_diff} shows the daily Average tone of the events across days. Whilst this data is less chaotic than the Dow Jones data, the moving average for 3 days was still plotted and is shown in Figure \ref{fig:avg_ma}. It is immediately apparent that the Average Tone is always negative, which is perhaps to be expected given the frosty relationship between the USA and China during the months of March and April. What the moving average shows, is that the Dow Jones follows a slightly similar path throughout time, only a few dats later, as the trough in the moving average for the Dow Jones difference occurs a few days after the trough in the average tone.

\begin{figure}[H]
	\centering
	\subfloat[GDELT Average Tone]{  \includegraphics[width=0.45\textwidth]{images/avgtone.png}\label{fig:avg}}
	\subfloat[GDELT Average Tone 3 Day Moving Average]{  \includegraphics[width=0.45\textwidth]{images/avgtone_ma.png}\label{fig:avg_ma}}\\
	\caption{GDELT Average Tone over time and Moving Average}
	\label{fig:avg_tone_diff}
\end{figure}

Figure \ref{fig:gs_diff} shows the daily average of the Goldstein Scale of events across time, and as with the previous two plots, the 3 day moving average is shown in Figure \ref{fig:gs_ma}. The Goldstein Score, like the Dow Jones difference, appears to be chaotic, shifting from positive to negative fairly frequently. The moving average shows that there appears to be a peak halfway through the time period, followed by more fluctuation. 

\begin{figure}[H]
	\centering
	\subfloat[GDELT Goldstein Scale]{  \includegraphics[width=0.45\textwidth]{images/goldsteinscore.png}\label{fig:gs}}
	\subfloat[GDELT Goldstein Scale 3 Day Moving Average]{  \includegraphics[width=0.45\textwidth]{images/goldsteinscore_ma.png}\label{fig:gs_ma}}\\
	\caption{GDELT Goldstein Scale over time and Moving Average}
	\label{fig:gs_diff}
\end{figure}

\subsection{Topic Modelling}
\subsubsection{TF-IDF}
The top Tf-idf terms are shown for the USA China data across the corpus in Figure \ref{fig:tfidfusachina}. This data was achieved by taking the average of the tfidf values across the entire dataset. The values are slightly lower, as there will be lots of tf-idf values of 0, where there are documents where the word doesn't exist. The 0 values of the Tf-Idf are included in the mean calculations as the parsing process for URLs is not perfect, and the Tf-Idf values can be non representative for some values if it was one of only 1 value present in the data.

\begin{figure}[H]
	\centering
	\includegraphics[width=0.49\textwidth]{Images/usa_stem_tfidf.png}
	\caption{Plot of the top 15 words which used TF-IDF in the USA/China specific data}
	\label{fig:tfidfusachina}
\end{figure}

Perhaps as expected, the most important and often occurring words are China and coronavirus.  Amongst the top words are also US, and Trump, along with variations of China and covid-19 and references to the pandemic. This is most interesting as a result, as something which no one had heard of prior to January/February dominated the news in March and April.

One of the other main words which pops up is html. This is most likely as a result of the fact that most of the URLs end with `.htm` or `.html`, and during the parsing html gets treated as a commonly occurring word. It was not removed as there could be legitimate stories which have the word in them. 

For the top values, the distribution of the tf-idf values across documents was calculated, excluding the 0 values. This is shown in Figure \ref{fig:tfidfdist}. The distributions are different to each other, but both follow a similar pattern in having a centre of the distribution be around a Tf-IDF value of around 0.25. One of the notable exceptions to this is the word `ar`, which is another error as a result of parsing. World and news both have a spike later on, but that is most likely due to the smaller sample size.

\begin{figure}[H]
	\centering
	\includegraphics[width=0.49\textwidth]{Images/usa_tfidf_top_distribution.png}
	\caption{Distribution of the TF-IDF values across documents of the top 15 words (excluding documents where the Tf-IDF value was 0)}
	\label{fig:tfidfdist}
\end{figure}

\subsubsection{LDA}
\paragraph{Initial LDA}
Initially a topic model was run on one day's worth of Event data. The day was the 1st of November 2019. This was done as a reference point before running the LDA models on the specific USA/China chosen data.

There were two different topic models tried. Firstly one model with 3 topics was tried and then one model with 5 topics. The top words from each model are shown in word cloud format in Figures \ref{fig:single3wc} and \ref{fig:single5}. Alongside the word cloud, for each topics, the weight and the word count of the top words was also calculated and plotted. 
	
\begin{figure}[H]
	\centering
	\subfloat[Word Cloud 3 Topics]{  \includegraphics[width=0.8\textwidth]{images/single/word_cloud_single_3_topics.png}\label{fig:single3wc}}
\end{figure}
\begin{figure}[H]
	\centering
	\ContinuedFloat
	\subfloat[Word Weights 3 topics]{  \includegraphics[width=0.8\textwidth]{images/single/word_weights_single_3_topics.png}\label{fig:single3ww}}\\
	
	\caption{Single Day Word Clouds for 5 topics}
	\label{fig:single3}
\end{figure}

Examining the word clouds for the model with three topics, there aren't any clear topics which are apparent. Topic 2 could vaguely be about the impeachment process for Donald Trump, Topic 0 appears to be focused on police brutality as a topic, and Topic 1 could broadly be referred to in terms of international news. Examining the Word importances, the main theme across topics ia that the word count is not the same as the word importance, in that some words have much higher occurrences, but lower weights and vice versa.

\begin{figure}[H]
	\centering
	\subfloat[Word Cloud 5 Topics]{  \includegraphics[width=0.8\textwidth]{images/single/word_cloud_single_5_topics.png}\label{fig:single5wc}}\\
	\subfloat[Word Weights 5 topics]{  \includegraphics[width=1\textwidth]{images/single/word_weights_single_5_topics.png}\label{fig:single5ww}}\\
	
	\caption{Single Day Word Clouds for 5 topics}
	\label{fig:single5}
\end{figure}	

Examining the 5 topic model, the topics are much closer together, its very difficult to find a central topic for each topic, topic 3 could potentially be about police and court information, but aside from that there does not appear to any coherency otherwise, with words like Trump being in multiple topics and international countries spread across topics. The word importance and weight plot also doesn't reveal anything new, like the previous model, the word's count is not related to the importance and perhaps expectedly, the words in the topics are not related to each other.

\paragraph{USA/China Data}
A similar procedure was used for the USA/China data, but models with 2, 3, and 4 topics each were tried. The word clouds of the results and the subsequent word importances to each topic are shown in Figures \ref{fig:usa2}, \ref{fig:usa3}, and \ref{fig:usa2}. 
\begin{figure}[H]
	\centering
	\subfloat[USA/China Word Cloud 2 Topics]{  \includegraphics[width=0.6\textwidth]{images/uschina/word_cloud_usa_2_topics.png}\label{fig:us2wc}}\\
	\subfloat[USA/China Word Weights 2 topics]{  \includegraphics[width=0.8\textwidth]{images/uschina/word_weights_usa_2_topics.png}\label{fig:us2ww}}\\
	
	\caption{USA/China Word Clouds for 2 topics}
	\label{fig:usa2}
\end{figure}
Examining the first LDA model which had 2 topics on the USA China data, the themes are very similar. Words related to the pandemic, and words such as China and Trump appear in both topics, which suggest the model hasn't been effective in differentiating between the topics effectively. Looking at the word weights in Figure \ref{fig:us2ww}, for all of the words, the weights are all higher than the word counts. The highest word weights are Trump, Coronavirus, China and China, Coronavirus, and `u`. `U` appears appears to be another issue with the parsing.  
\begin{figure}[H]
	\centering
	\subfloat[USA/China Word Cloud 3 Topics]{  \includegraphics[width=0.6\textwidth]{images/uschina/word_cloud_usa_3_topics.png}\label{fig:us3wc}}\\
	\subfloat[USA/China Word Weights 3 topics]{  \includegraphics[width=0.8\textwidth]{images/uschina/word_weights_usa_3_topics.png}\label{fig:us3ww}}\\
	
	\caption{USA/China Word Clouds for 3 topics}
	\label{fig:usa3}
\end{figure}

Looking the the three topic model, the topics are even closer together than the two topic model. The main words as before appear in all of the topics, suggesting the topic model hasn't separated any topics well. The word weights are also similar to the two topic model. One of the differences between the two topic and the three topic model is the 3rd topic weights are noticeably smaller than the first and second topics. 

\begin{figure}[H]
	\centering
	\subfloat[USA/China Word Cloud 4 Topics]{  \includegraphics[width=0.6\textwidth]{images/uschina/word_cloud_usa_4_topics.png}\label{fig:us4wc}}\\
	\subfloat[USA/China Word Weights 4 topics]{  \includegraphics[width=0.8\textwidth]{images/uschina/word_weights_usa_4_topics.png}\label{fig:us4ww}}\\
	
	\caption{USA/China Word Clouds for 4 topics}
	\label{fig:usa4}
\end{figure}

The 4 topic model behaves in a similar manner to the previous 2 topic models. the main words are split across the 4 topics with no real distinction between the topics present. 

\subsubsection{K Means}
Similar to the LDA models, the K-means was tried with 2, 3, and 4 clusters. The word cloud of the top words are shown in Figures \ref{fig:wck2}, \ref{fig:wck3}, and \ref{fig:wck4}. Alongside the word clouds, the clusters were decomposed into 2 and 3 dimensions by both Principal Component Analysis (PCA) and T-distributed Stochastic Neighbour Embedding (T-SNE). This was to see whether the clusters had been successful in separating the data at any level. This is shown for the different number of cluster in Figures \ref{fig:k2pca}, \ref{fig:k3pca}, and \ref{fig:k4pca} respectively. Furthermore, the Mahalanobis and Euclidean distances were plotted for all of the points associated with a cluster. This is shown in Figures \ref{fig:distk2}, \ref{fig:distk3}, and \ref{fig:distk4} respectively.
\paragraph{Clustering with k=2}
\begin{figure}[H]
	\centering
	\includegraphics[width=0.8\textwidth]{images/kmeans_word_cloud_k=2.png}
	\caption{Word Cloud for k=2 clusters}
	\label{fig:wck2}
\end{figure}
Examining the word cloud created for a two cluster model, a similar picture appears as with the LDA models. There are several words which are close to both of the clusters, which, in a similar fashion to the LDA model, mean that any topics present are not being differentiated. This is evident in Figure \ref{fig:k2pca}, in which both T-SNE and PCA in both 2 and 3 dimensions show that the clusters are not distinct from each other. Interestingly, both the 2d and 3D PCA plots, Figures \ref{fig:pca2k2} and \ref{fig:pca2k3}, appear to show clean boundaries, however, it is not the most obvious of cluster definitions.

\begin{figure}[H]
	\centering
	\subfloat[2D PCA k=2]{  \includegraphics[width=0.45\textwidth]{images/kmeans_2d_pca_k=2.png}\label{fig:pca2k2}}
	\subfloat[3D PCA k=2]{  \includegraphics[width=0.45\textwidth]{images/kmeans_3d_pca_k=2.png}\label{fig:ts2k2}}\\
	\subfloat[2D T-SNE k=2]{  \includegraphics[width=0.45\textwidth]{images/kmeans_2d_tsne_k=2.png}\label{fig:pca3k2}}
	\subfloat[3D T-SNE k=2]{  \includegraphics[width=0.45\textwidth]{images/kmeans_3d_tsne_k=2.png}\label{fig:ts3k2}}\\
	\caption{Decompositions of the clusters in 2 and 3 dimensions using PCA and T-SNE for k=2}
	\label{fig:k2pca}
\end{figure}
Examining the Mahalanobis distances for the two clusters, there doesn't appear to be any noticeable pattern, the distributions are slightly bimodal, compared to the Euclidean distances, which are much more normally distributed. The majority of points are associated with cluster 0, with the rest going to Cluster 1.
\begin{figure}[H]
	\centering
	\subfloat[Mahalanobis Distance]{  \includegraphics[width=0.45\textwidth]{images/kmeans_mahalanobis_distance_k=2.png}\label{fig:mhk2}}
	\subfloat[Euclidean Distances]{  \includegraphics[width=0.45\textwidth]{images/kmeans_euclidean_distance_k=2.png}\label{fig:euk2}}\\
	
	\caption{Cluster Distances (Mahalanobis and Euclidean) for k=2 clusters}
	\label{fig:distk2}
\end{figure}

\paragraph{Clustering when k=3} 
Examining the word cloud when there are 3 clusters, all of the clusters are similar to the k=2, in that they have very similar topics related to China and the Covid 19 Pandemic. Examining the decomposition plots, the clusters are close together, and the cluster definitions do not appear to be definitively useful. There are two smaller clusters present, which potentially could be clusters, but when examined, they represented a mixture of parsing errors from URLs, and numbers and letter combinations left over from URL parsing.  
\begin{figure}[H]
	\centering
	\includegraphics[width=0.8\textwidth]{images/kmeans_word_cloud_k=3.png}
	\caption{Word Cloud for k=3 clusters}
	\label{fig:wck3}
\end{figure}

\begin{figure}[H]
	\centering
	\subfloat[2D PCA k=3]{  \includegraphics[width=0.45\textwidth]{images/kmeans_2d_pca_k=3.png}\label{fig:pca2k3}}
	\subfloat[3D PCA k=3]{  \includegraphics[width=0.45\textwidth]{images/kmeans_3d_pca_k=3.png}\label{fig:ts2k3}}\\
	\subfloat[2D T-SNE k=3]{  \includegraphics[width=0.45\textwidth]{images/kmeans_2d_tsne_k=3.png}\label{fig:pca3k3}}
	\subfloat[3D T-SNE k=3]{  \includegraphics[width=0.45\textwidth]{images/kmeans_3d_tsne_k=3.png}\label{fig:ts3k3}}\\
	\caption{Decompositions of the clusters in 2 and 3 dimensions using PCA and T-SNE for k=3}
	\label{fig:k3pca}
\end{figure}
The Mahalanobis distance, as with the k=2, appear to be bimodal with a peak between 35 and 40, and a second peak above 50. The Euclidean distances are closer to being normally distributed. The majority of points are associated with cluster 2, with cluster 0 being the second biggest, and cluster 1 being the smallest.  


\begin{figure}[H]
	\centering
	\subfloat[Mahalanobis Distance]{  \includegraphics[width=0.45\textwidth]{images/kmeans_mahalanobis_distance_k=3.png}\label{fig:mhk3}}
	\subfloat[Euclidean Distances]{  \includegraphics[width=0.45\textwidth]{images/kmeans_euclidean_distance_k=3.png}\label{fig:euk3}}\\
	
	\caption{Cluster Distances (Mahalanobis and Euclidean) for k=3 clusters}
	\label{fig:distk3}
\end{figure}

\paragraph{Clustering when k=4}
Looking at the word clouds for 4 clusters It appears to be similar to both k=2 and k=3, in that all of the main words have been shared across the clusters. Cluster 2 has slightly more of the parsing errors associated with it, and thus is the more scattered of the clusters, and cluster 3 is slightly more focused on the pandemic in the US with words such as Test, and White House.

The decompositions are similar to the previous cluster results, especially in the PCA which suggests that boundaries between the clusters are very strong, but the `cluster` each cluster creates is not representative.
\begin{figure}[H]
	\centering
	\includegraphics[width=0.8\textwidth]{images/kmeans_word_cloud_k=4.png}
	\caption{Word Cloud for k=4 clusters}
	\label{fig:wck4}
\end{figure}


\begin{figure}[H]
	\centering
	\subfloat[2D PCA k=4]{  \includegraphics[width=0.45\textwidth]{images/kmeans_2d_pca_k=4.png}\label{fig:pca2k4}}
	\subfloat[3D PCA k=4]{  \includegraphics[width=0.45\textwidth]{images/kmeans_3d_pca_k=4.png}\label{fig:ts2k4}}\\
	\subfloat[2D T-SNE k=4]{  \includegraphics[width=0.45\textwidth]{images/kmeans_2d_tsne_k=4.png}\label{fig:pca3k4}}
	\subfloat[3D T-SNE k=4]{  \includegraphics[width=0.45\textwidth]{images/kmeans_3d_tsne_k=4.png}\label{fig:ts3k4}}\\
	\caption{Decompositions of the clusters in 2 and 3 dimensions using PCA and T-SNE for k=4}
	\label{fig:k4pca}
\end{figure}
In line with the previous clusters, the Mahalanobis distance is also bimodal. The Euclidean distance as before appears to be more normal. In this cluster, Cluster 2 gets the majority of the words, followed by Cluster 3 and 0, and then Cluster 1. 

\begin{figure}[H]
	\centering
	\subfloat[Mahalanobis Distance]{  \includegraphics[width=0.45\textwidth]{images/kmeans_mahalanobis_distance_k=4.png}\label{fig:mhk4}}
	\subfloat[Euclidean Distances]{  \includegraphics[width=0.45\textwidth]{images/kmeans_euclidean_distance_k=4.png}\label{fig:euk4}}\\
	
	\caption{Cluster Distances (Mahalanobis and Euclidean) for k=4 clusters}
	\label{fig:distk4}
\end{figure}

\subsection{Modelling Stock Prices}
The table of accuracy predictions for the different models is shown in Table \ref{table:modaccuracy}. Several models had similar accuracy scores, though it should be noted that the predictions were on a very small dataset, so even one or two correct or incorrect predictions could sway the result considerably. 

The manually lagged models appeared to be the least best, in these models 5 days worth of previous days were added as 5 features to the data, and no other averaging methodology applied. Interestingly, the random forests classifiers appeared to do well with whatever data structure was used. Furthermore, models which applied moving window calculations did well, with on average the exponentially smoothed models doing the best.

With all of that taken into consideration, the best model was picked on the basis of being exponential and the random forests, and was found to be the exponential random forest.
\begin{table}[H]
	\centering 
\begin{tabular}{lr}

	Classifier &  Accuracy \\
	\hline
	Reference Classifier with Previous Day & 0.750\\
	\hline
                                  Naive Bayes &     0.750 \\
Logistic Regression &     0.750 \\
Averaged Random Forests &     0.750 \\
Gaussian Averaged Random Forests &     0.750 \\
Exponential Naive Bayes &     0.750 \\
Exponential Logistic Regression &     0.750 \\
Exponential Random Forests &     0.750 \\
Random Forests &     0.625 \\
Support Vector Machine &     0.625 \\
Averaged Naive Bayes &     0.625 \\
Averaged Logistic Regression &     0.625 \\
Averaged Support Vector Machine &     0.625 \\
Gaussian Averaged Naive Bayes &     0.625 \\
Gaussian Averaged Logistic Regression &     0.625 \\
Gaussian Averaged Support Vector Machine &     0.625 \\
Exponential Support Vector Machine &     0.625 \\
Manually 4 day lagged Random Forests &     0.625 \\
Neural Network & 0.500 \\
Manually 4 day lagged Naive Bayes &     0.500 \\
Manually 4 day lagged Logistic Regression &     0.500 \\
Manually 4 day lagged Support Vector Machine &     0.375 \\
	

\end{tabular}

	\caption{Table of Model Accuracy Results}
		\label{table:modaccuracy}
\end{table}

The best model's algorithm was used to train another model which just used the previous day's difference, as a reference to see if the model actually provided any improvement. It found that the accuracy improved considerably on the May June Data, with the reference model providing 50\% accuracy, and the exponentially smoothed random forest providing 62.5\% accuracy, which on the face of it is a significant improvement. 

The predictions are shown in figure \ref{fig:modelpred}, where the magnitude of the differences has been plotted, alongside whether the best model was correct in predicting a positive or negative shift (blanks are weekends or days when the stock market didn't operate). There does not appear to be a time based prediction component, the model is just as likely to predict correctly or incorrectly regardless of the time that it is predicting on. Interestingly, the model does not appear to be very good at predicting for large stretches correctly, it correctly predicts for a day or two, and then has incorrect predictions for a similar length of interval. The exception to this is when halfway through the prediction, the model correctly predicts the spike and the subsequent fall and recovery. 

The distribution of magnitude of the positive and negative daily differences is plotted in Figure \ref{fig:distmodel}.  It is apparent that the model is better at predicting smaller differences in the stock market. the majority of incorrect predictions come from larger negative values and slightly from the positive jumps. There are significantly fewer incorrect predictions when the market jumps between -500 and +500 points. Interestingly, there is a small bump at a large positive value of +1000, suggesting the model was good at predicting large positive values. 

\begin{figure}[H]
	\centering
	\includegraphics[width=0.8\textwidth]{images/model_pred_time.png}
	\caption{The Dow Jones daily changes in prediction dataset, and whether the model predicted the the result correctly or not)}
	\label{fig:modelpred}
\end{figure}


\begin{figure}[H]
	\centering
	\includegraphics[width=0.8\textwidth]{images/dist_of_model.png}
	\caption{The distribution of the daily differences between open and close prices where the model predicted correctly vs predicted incorrectly)}
	\label{fig:distmodel}
\end{figure}






\section{Discussion}
The broader purpose of topic modelling would be to build a topic model which would be able to categorise new and upcoming geopolitical risks, and thus be able to filter through only the relevant information. In the context of using GDELT, it would mean filtering out the headlines to only use the tone and Goldstein scale. This would mean that the topic model would not only have to be able to curate existing risks adequately, but also be able to catch new emerging risks too.

After performing the cluster analysis, and modelling, there were several things which became apparent. Firstly, the topic models are good at finding the main topics. Both the clusters and the LDA topic models were able to find what would be considered the `main` words in the topic, in that if a human were to look for the most important words, this is what they would be. Furthermore, if the Tf-IDF results were examined, the results were surprisingly insightful into the data. The results were mostly coherent and representative of what a human would retrospectively brainstorm for the time period of the training data.

However, one of the main concerns for both the LDA model and the clustering algorithm is the lack of separability. In both of the topic modelling attempts it appeared that there was only ever one topic being detected, in that there was no separation between the words such as . This could perhaps be a result of there not being at least two topics present in the data. However, comparing the single day data and the USA China data, neither set of word clouds strongly show a prevailing topic, though both represent the data reasonably. This in itself is extremely promising, as the corpus and documents themselves were composed of URLs which were brute force parsed, which in turn meant that there were a lot of `garbage` documents, the result of parsed URLs where the headline wasn't coherent or even not there.

The main concern with the topic models is the fact that existing topic models cannot be used effectively in terms of filtering new information, and thus cannot detect new topics. One of the biggest issues with LDA models is that the documents themselves have been generated from a specific set of topics, which means that if the word being filtered didn't exist in the corpus there is no probability/closeness metric that can generated for each topic (at least in the implementation that was used). This means that one of the only ways a topic model could be used for new data is if an entire dictionary was used in the corpus alongside the corpus. This in turn would create two other problems. Firstly, it would disrupt the meaning of the topics as there would be an excess of unrelated words to the documents. The second issue with this approach is more prohibitive. New words and topics appear constantly in the media. For example, a topic model in November would not have been able to predict the importance of topics associated with Coronavirus. This is especially problematic for geopolitical modelling, as often it would be events appearing out of the blue which cause the most amount of disruption.  

A similar problem exists for K Means. The KMeans algorithm for words in this instance uses the TFIDF vectorization to build clusters. This relies upon all of the words existing in corpus in a manner where the most important and common words appear most frequently. This means inherently the weighting is designed for a fully complete corpus, and not one which is being used to analyse new information. One of the main differences between the K Means algorithm and the LDA algorithm is that for K means it is possible to get a quantitative number for a word being close to a particular cluster, as Tf-IDF weightings can be calculated by inserting an offset for words which aren't in the corpus. However, this means that the distances to the cluster centres for such words are much less useful, as there is no way for an existing cluster model to tell apart a new rising topic, for example for a pre-covid trained model, words related to coronavirus or covid, from information which is irrelevant or noise, as they would both have the same calculated TF-IDF value with regards to the dataset.

As such this was the main reason the topic model was not used for filtering for a secondary dataset and why this approach in this fashion cannot be used accurately. 

Looking at the model results, it is difficult to say with any certainty what the best model is. The models were only trained on 2 months worth of data, which isn't sufficient to conclusively identify which model is the `best`, or at least most accurate. However, there are certain features of the models which become apparent, generally weighted models deal well with the data, the manually lagged non weighted model was certainly the worst off of all of the models which were tried. This does back up the theory that the Average Tone and the Goldstein Score may influence stock prices several days down the line, with the most important day being the last day in the moving window.

One of the other issues with this prediction interval is that the data restricts what the model has learned on. There is no guarantee that this would be representative over the much longer term, which would be more useful for investors.

One of the pertinent issues with prediction such as this is the importance to remember that in the long term the stock market always grows, thus this strategy would have to prove better than just investing in the long term growth of the stock market. 






\section{Conclusions}
I have conclusions




\section{Further Work}

There is a large amount of further work which can be explored. One of the most apparent extensions of this would be to try these methods on more data. GDELT is a large resource, and a vast amount of data is available to be explored and tried, from data from multiple other countries, to using more historical long term data.


One of the biggest challenges with this work is that it only works on a single word basis. All of the clusters etc are based on single words, and not phrases, or content. This means there would still have to be a large amount of manual work required when using this approach, to ensure that the algorithms do not end up working with incoherence. The principle of garbage in garbage out also applies here, there is a substantial amount of preprocessing required to ensure that you're feeding the algorithm useful information, and even after that,

Another further exploration could be an attempt to predict the stock prices directly, this work only aimed to predict changes in stock, and as a result this doesn't take magnitude of change into consideration, a large stock fall is the same as a small stock fall, but investor reactions may be different if it is only a minor fall in the market, and 
\appendix



\begin{appendices}
	\section{Phrases to Test K-Means}
	The phrases with which the K Means algorithm was tested are shown below:
	\label{phrases}
\renewcommand\labelitemi{---}
	\begin{itemize}
\item 			 coronavirus hits remote utah
		\item scottish wind power success 
\item 		trump passes bill
		\item covid
		\item aboriginal peoples australia complain
		\item nikkei closes 90 points down
	\end{itemize}

	\section{K Means Clustering k=3}
	\label{k3}
	\begin{figure}[H]
		\centering
		\includegraphics[width=0.8\textwidth]{images/kmeans_word_cloud_k=3.png}
		\caption{Word Cloud for k=3 clusters}
		\label{fig:wck3}
	\end{figure}
	
	\begin{figure}[H]
		\centering
		\subfloat[2D PCA k=3]{  \includegraphics[width=0.45\textwidth]{images/kmeans_2d_pca_k=3.png}\label{fig:pca2k3}}
		\subfloat[3D PCA k=3]{  \includegraphics[width=0.45\textwidth]{images/kmeans_3d_pca_k=3.png}\label{fig:pca3k3}}\\
		\subfloat[2D T-SNE k=3]{  \includegraphics[width=0.45\textwidth]{images/kmeans_2d_tsne_k=3.png}\label{fig:ts2k3}}
		\subfloat[3D T-SNE k=3]{  \includegraphics[width=0.45\textwidth]{images/kmeans_3d_tsne_k=3.png}\label{fig:ts3k3}}\\
		\caption{Decompositions of the clusters in 2 and 3 dimensions using PCA and T-SNE for k=3}
		\label{fig:k3pca}
	\end{figure}
	
	\begin{figure}[H]
		\centering
		\subfloat[Mahalanobis Distance]{  \includegraphics[width=0.45\textwidth]{images/kmeans_mahalanobis_distance_k=3.png}\label{fig:mhk3}}
		\subfloat[Euclidean Distances]{  \includegraphics[width=0.45\textwidth]{images/kmeans_euclidean_distance_k=3.png}\label{fig:euk3}}\\
		
		\caption{Cluster Distances (Mahalanobis and Euclidean) for k=3 clusters}
		\label{fig:distk3}
	\end{figure}
	
	\begin{figure}[H]
		\centering
		\includegraphics[width=0.8\textwidth]{images/words_kmeans_mahalanobis_distance_k=3.png}
		\caption{Log of Mahalanobis Distances for Clusters and Selected Phrases for k=3 clusters}
		\label{fig:wordsk3}
	\end{figure}
	\section{K Means Clustering k=4}
	\label{k4}
	\begin{figure}[H]
		\centering
		\includegraphics[width=0.8\textwidth]{images/kmeans_word_cloud_k=4.png}
		\caption{Word Cloud for k=4 clusters}
		\label{fig:wck4}
	\end{figure}
	
	
	\begin{figure}[H]
		\centering
		\subfloat[2D PCA k=4]{  \includegraphics[width=0.45\textwidth]{images/kmeans_2d_pca_k=4.png}\label{fig:pca2k4}}
		\subfloat[3D PCA k=4]{  \includegraphics[width=0.45\textwidth]{images/kmeans_3d_pca_k=4.png}\label{fig:pca3k4}}\\
		\subfloat[2D T-SNE k=4]{  \includegraphics[width=0.45\textwidth]{images/kmeans_2d_tsne_k=4.png}\label{fig:ts2k4}}
		\subfloat[3D T-SNE k=4]{  \includegraphics[width=0.45\textwidth]{images/kmeans_3d_tsne_k=4.png}\label{fig:ts3k4}}\\
		\caption{Decompositions of the clusters in 2 and 3 dimensions using PCA and T-SNE for k=4}
		\label{fig:k4pca}
	\end{figure}
	
	\begin{figure}[H]
		\centering
		\subfloat[Mahalanobis Distance]{  \includegraphics[width=0.45\textwidth]{images/kmeans_mahalanobis_distance_k=4.png}\label{fig:mhk4}}
		\subfloat[Euclidean Distances]{  \includegraphics[width=0.45\textwidth]{images/kmeans_euclidean_distance_k=4.png}\label{fig:euk4}}\\
		
		\caption{Cluster Distances (Mahalanobis and Euclidean) for k=4 clusters}
		\label{fig:distk4}
	\end{figure}
	
	
	\begin{figure}[H]
		\centering
		\includegraphics[width=0.8\textwidth]{images/words_kmeans_mahalanobis_distance_k=4.png}
		\caption{Log of Mahalanobis Distances for Clusters and Selected Phrases for k=4 clusters}
		\label{fig:wordsk4}
	\end{figure}
\end{appendices}


\bibliographystyle{acm}
\bibliography{ref}
\end{document}
