\section{GDELT}

The Global Database for Events Language and Tone was the basis for this dissertation. This is a database which stores a record of every broadcast, print, and web news published since 1979, in over 100 languages from all parts of the world. It contains over a quarter of a billion georeferenced records. 

The records are stored across multiple tables accessible either directly, or through Google Big Query. The main table which was the source of the data in this dissertation was the Events table. This table maintains the record of all events, along with cursory information about each event, such as which countries it happened in or between, the dates, and the URLs of the first news article this appeared in.

The interesting part of the events table is the source URL, the Average Tone and the Goldstein Scale. This dissertation and prototyping works on the basis of the headline if it is present in the source URL. The reason for this is that it would be infeasible to retrieve each article and process the documentation, as each article and site would have a completely different html structure, and automating the parsing would be almost impossible. Furthermore, some articles may be beyond paywalls, which would be difficult to reach with conventional web scraping methods such as wget.

For each event, there are two columns of note, the Average Tone and the Goldstein Scale. The Average Tone is a measure of the sentiment of all the news media which describes the specific event. It is on a scale of -100 (Extremely Negative) to +100 (Extremely Positive). The most common values observed are usually between -10 and 10, and 0 indicated a neutral tone. 

The Goldstein Scale is the main metric by which geopolitical modelling occurs, and was initially defined as the Conflict-Cooperation Scale for Work Environment Impact Scale (WEIS) events data \cite{goldstein1992conflict}. This is a scale between -10 and +10 which captures the potential impact an event will have on the stability of the country, with -10 being the worst impact an event can have and +10 being the most positive impact an event can have. It should be noted that these numbers are predefined and do not take into consideration the extent to which an even occurs, for example event type 071, Extending economic aid; give, buy, sell, or borrow, is given a score of +7.4, regardless of whether it is £1 of aid given or £1 million. 

GDELT has been used previously for stock predictions in the past \cite{memari2017predicting} \cite{alamro2019predicting}, however, it is used much more commonly used for predicting and analysing geopolitical events themselves \cite{qiao2017predicting} \cite{yonamine2013predicting}. 