\section{GDELT}
\label{gdelt}

The Global Database for Events Language and Tone was the basis for this dissertation. This is a database which stores a record of every broadcast, print, and web news published since 1979, in over 100 languages from all parts of the world. 

The records are stored across multiple database tables accessible either directly, or through SQL in Google Big Query. The main table which was the source of the data in this dissertation was the Events table. This table maintains the record of all events, along with cursory information about each event, such as which countries it happened in or between, the dates, and the URLs of the first news article this appeared in. The events table contains over a quarter of a billion georeferenced records each referring to an event which occurred since the 1st of January 1979. 

For each event, there are three variables of note for this dissertation, the source URL, the Average Tone and the Goldstein Scale. The source URLs were the URLs of the first headline which covered the event. The topic models were built on these headlines of the articles extracted from the source URLs, where present. The Average Tone is a measure of the sentiment of all the news media which describes the specific event. It is on a scale of -100 (Extremely Negative) to +100 (Extremely Positive). The most common values observed are usually between -10 and 10, and 0 indicated a neutral tone. 

The Goldstein Scale is the main metric by which geopolitical modelling occurs, and was initially defined as the Conflict-Cooperation Scale for Work Environment Impact Scale (WEIS) events data \cite{goldstein1992conflict}. This is a scale between -10 and +10 which captures the potential impact an event will have on the stability of the entity in question (usually a country), with -10 being the worst impact an event can have and +10 being the most positive impact an event can have. It should be noted that these numbers are predefined and do not take into consideration the extent to which an event occurs, for example event type 071, Extending economic aid; give, buy, sell, or borrow, is given a score of +7.4, regardless of whether it is £1 of aid given or £1 million. 

GDELT has been used for stock predictions in the past \cite{memari2017predicting} \cite{alamro2019predicting}, however, it is used much more commonly for predicting and analysing geopolitical events themselves \cite{qiao2017predicting} \cite{yonamine2013predicting}. 