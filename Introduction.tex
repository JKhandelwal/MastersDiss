\section{Introduction}
Using geopolitical news to predict stock prices would be of great interest to investors. If, by mining geopolitical news predictions are possible and accurate with regards to the direction the stock market will go, it would mean investors will be able to not only not \textit{lose} money, e.g. by selling in a bear market before the price crashes, but in certain circumstances, be able to make money, e.g. buying in a bull market before the price increases.

This dissertation was focused around using the Global Database of Events, Language, and Tone (GDELT) to build topic models which could be used to predict the changes in the stock market. The topic models and the GDELT data used was related to geopolitical information. The aim was to use GDELT data to build a topic model which could be used to filter news information for specifically geopolitical news. Then the geopolitical measure of the conflict cooperation Goldstein Scale for geopolitical events provided by GDELT would be used to try and predict information on the stock market.

The methodology behind building a topic model took news headlines from the GDELT data over a period of time, and then performed cluster analysis/topic curation on the headlines themselves, with the aim being to tease out the underlying topics, which would ideally be specifically focused on the geopolitical topics. The next step was to use the curated topic model to filter other news to end up with headlines specifically related to that geopolitical topic. Then a variety of models could be tested to aim to predict the stock market shifts.

The initial approach to building the topic models was to use existing topic modelling algorithms. The first algorithm used was the Latent Dirichlet Analysis on data from the GDELT Events Table, filtered to data which which was specific to events concerning the USA or China. The aim for this data would be to sort all of the words present into K topics. USA and China specific data was chosen for two reasons. Firstly it was to narrow down the dataset, as GDELT is a very large resource, and it would be simpler initially to run on a simpler dataset. The second reason for this is using geopolitical news would have to be targetted considerably to specific events in specific countries and specific markets. Over the time period from March to April, there were significant events which took place specifically between those two countries, and it was thought this would provide the best opportunity to find signal within the data. 

However, this approach would not allow filtering by calculating the distance between the topics and unseen news, thus a different model was used. The K Means clustering algorithm was used alongside the Term Frequency Inverse Document Frequency (TF-IDF) to build clusters from the words in the headlines. This would mean that the clusters would be built based on words and phrases which occur frequently. It was hoped those clusters would represent the words and the phrases which are relevant to geopolitical information, and thus it could be used. This approach also did not work as initially thought, as due to the TF-IDF vectorisation the algorithm was unable to perform filtering in a usable fashion as it did not filter relevant geopolitical news effectively from noise. Thus this was not used to filter news for the modelling approaches, instead the modelling was on the entirety of the selected USA China data, without any further filtering. 

The modelling worked on the basis of predicting shifts in the Dow Jones Industrial Average. This was chosen again because geopolitical risk prediction would be done on specific markets, and it was thought that any events concerning the USA and China would be reflected into the Dow Jones. When making predictive models, stock shift was used for predictions as opposed to predicting exact stock price. It would also be assumed that the effect of events would not be singular, and instead effect the market for several days. Thus several different models with different styles of lagged Goldstein Scale and Average Tone values were modelled against the stock shift. The best model picked was a Random Forests Classifier which was used to predict in the interval of May and June 2020, compared against a model which just used the information from the previous day to predict whether the stock price increased or decreased. 

This model accrued a higher accuracy of predicting the stock shift over the prediction interval, however there is a large scope for further modelling to explore both GDELT and topic modelling further, both in terms of the scale of the data being predicted, and different markets and modelling strategies. 

This report will cover relevant background literature in Sections 2 to 4, an overview of the experiments performed and results in chapters 6 and 7, and a discussion of the results in chapter 8.