\section{Existing Work}
\label{prior}
There exist several approaches to model and predict stock prices. These approaches use many different sources of data and varying modelling styles. However, there remains limited prior work in modelling stock prices using calculated Geopolitical risk, though there are limited approaches to calculating geopolitical risk on its own.

\subsection{Geopolitical risk}
The majority of approaches in modelling Geopolitical risk work on using news information along with manually defined terms of geopolitical risk. One of the most recent approaches was done by Dario Caldara and Matteo Iacoviello, where they created a Geopolitical risk index from a series of news articles \cite{caldara2018measuring}. 

This risk index was calculated by manually creating search terms which define geopolitical risk in several categories, and then over time calculate the percentage of articles of selected number of news media in which the search terms appear. This is then used to model the stock prices.

There are other risk measures present, Black Rock also have a publicly viewable index called the BlackRock Geopolitical Risk Indicator (BGRI) \footnote{https://www.blackrock.com/corporate/insights/blackrock-investment-institute/interactive-charts/geopolitical-risk-dashboard}, which works in a similar fashion where key words are selected which define geopolitical risk, followed by an application of a sentiment score from a list of predefined words, and then those two scores are combined to make the BGRI total score. Furthermore, there are also many other organisations which almost certainly will model Geopolitical Indices, such as Reuters, but will not disclose the exact modelling for proprietary purposes.

One of the major differences between the existing approaches in modelling geopolitical risk and the approach used in this dissertation, is here we incorporate the cooperation conflict scale provided in GDELT. GDELT also differs from the data sources used in existing modelling approaches. In existing approaches, only manually selected news media are used for the data, whereas GDELT does not restrict which news are stored and calculated, and thus has a much wider scope in terms of news and events stored. Furthermore, this approach attempts to use topic modelling to filter the news, as opposed to using manually defined search terms, in an attempt to make the prediction process more automated. 

\subsection{Modelling Stock Prices}
There is a large vested interest for people to aim to model the stock market effectively, as any investor with extra knowledge about where the market will go will be able to take a position most beneficial to them. Thus, since having accurate predictive models are so beneficial, there exist many models which aim to predict market changes.

The Efficient Markets Hypothesis asserts that current securities prices reflect all available information and expectations\cite{fama1960efficient}. This is related to random walk theory, which in finance literature is used to describe a price series where all price changes are random departures from earlier prices, which means that any specific day's price change is not related to the previous day's price and only on the news that is received on that day. Whilst this theory has become more controversial \cite{malkiel2003efficient}, it is accepted that stock prices are related to new news. This means that Geopolitical events which generate news, and by extension an index of risk which track them, should have a measurable effect on stock prices. For example, the risk would increase if a `bad` event occurs, such as an increase in international conflict, or a negative piece of news is published. Since the risk would be higher it could be assumed that stock prices would decrease as investors could be less likely to buy and more likely to sell which would cause the decrease in the price. If the reverse were to happen with the risk index, it could be assumed the opposite would happen in the market.

As stock market modelling has existed for many years there have been several esoteric factors used to model and predict stock prices aside from the news, such as the weather on Wall Street \cite{saunders1993stock}. For most of the non quantitative data, sentiment analysis is commonly used to help predict stock prices, where natural language processing is used to gain a measure of tone in the data being used. The tone is a quantitative measure of how positive or negative the text in the data being used. This is then used to model the price changes. A number of studies shows that large amounts of data on social medias such as Facebook and Twitter can be useful for forecasting economic indicators such as the stock market \cite{bollen2011twitter} \cite{arias2014forecasting}. 

There are many different machine learning and statistical models used for modelling stock prices. One of the simplest approaches to stock price modelling is to perform autoregressive modelling on purely the time series data for stock prices. In this approach, the only thing which is used to predict the prices are the past values of the stock. One common approach is to use an autoregressive integrated moving average model (ARIMA) \cite{ariyo2014stock}. This is part of a set of models where the response variable, Y, is an ordered time series, where the Y for any given time is dependent on and calculated only from previous values in the time series. 

When predicting stock prices from variables other than the previous prices, many different machine learning algorithms are used. Aside from classical logistic regression models, these include SVMs \cite{cao2003support}, Random Forests \cite{khaidem2016predicting}, and Neural Networks \cite{egeli2003stock}, along with Bayesian approaches such as Naive Bayes \cite{khedr2017predicting}. These models can be used to predict both the stock prices, with a numeric response variable, and binary stock shift \cite{nguyen2015sentiment}, with a class response.  

There is another concern with modelling on the stock market. When predicting any stock market, one of the major issues which arises is that in the long term the stock market always grows, for the S and P for example there is an annualised growth of on average 9.5\% \cite{sp} without the affect of inflation factored in. Thus any investor investing over a sufficiently long period of time will nearly always make money. The aim then, is to build models which would give better results than the normal stock market rise, and thus any strategy or predictions by any model would have to prove and provide better returns than just investing in the long term growth of the stock market.

A mixture of these algorithms were used alongside the GDELT algorithm to try and predict stock shift. 

