\section{Existing Work}

There exist many approaches, and many different models to aim to predict stock prices. These approaches use many different sources of data, and many different modelling styles to predict stock prices. It is common to use sentiment analysis to predict stock prices for example. However, there remains limited prior work in modelling stock prices using calculated Geopolitical risk, though there are limited approaches to calculating geopolitical risk on its own.

\subsection{Geopolitical risk}
The majority of approaches in modelling Geopolitical risk work on using news information along with manually defined terms of geopolitical risk. One of the most recent approaches was done by Dario Caldara and Matteo Iacoviello, where they created a Geopolitical risk index from a series of news articles \cite{caldara2018measuring}. 

This risk index was calculated by manually creating search terms which define geopolitical risk in several categories, and then over time calculate the percentage of articles of selected number of news media in which the search terms appear. This is then used to model against the stock prices.

There are other risk measures present, Black Rock also have a publicly viewable index called the BlackRock Geopolitical Risk Indicator (BGRI) \footnote{https://www.blackrock.com/corporate/insights/blackrock-investment-institute/interactive-charts/geopolitical-risk-dashboard}, which works in a similar fashion where key words are selected which define geopolitical risk, followed by an application of a sentiment score from a list of predefined words, and then those two scores are combined to make the BGRI total score. Furthermore, there are also many other organisations which almost certainly will model Geopolitical Indices, such as Reuters, but will not disclose the exact modelling for proprietary purposes.

One of the major differences between the existing approaches in modelling geopolitical risk and the approach this dissertation attempted to use the cooperation conflict scale provided in GDELT. The main differences are that GDELT does not restrict which news media are used for reference, instead using more than any other . Furthermore, this approach attempts to use topic modelling to filter the news, as opposed to using manually defined search terms, in an attempt to make the prediction process more automated. 

\subsection{Modelling Stock Prices}
There is a large vested interest for people to aim to model the stock market effectively, as any investor with extra knowledge about where the market will go will be able to take a position most beneficial to them. Thus, there are a large number of models which exist and are used which aim to predict the changes in the markets. 

The efficient markets hypothesis claims that markets are rational and act upon indicators which can be predicted. This means that Geopolitical events, and by extension an index of risk should have a measurable effect on stock prices, in that, if the risk increases if an bad event occurs or a negative piece of news is published, it would be assumed that stock prices would decrease as people would be more likely to sell which would cause the decrease in the price, and vice versa.

As stock market modelling has existed for many years there have been many esoteric factors used to model and predict stock prices aside from the news, such as the weather on Wall Street \cite{saunders1993stock}. For most of the non quantitative data, sentiment analysis is commonly used to help predict stock prices, where natural language processing is used to gain a measure of tone in data being used, and from that, the changes in the price is modelled. This is similar to the average tone measured by the GDELT. However, there is limited prior work using information at the scale of which GDELT provides. However, a number of studies shows that large amounts of data on social medias such as Facebook and Twitter can be useful for forecasting economic indicators such as the stock market \cite{bollen2011twitter} \cite{arias2014forecasting}. 

With regards to the models themselves, there are many different machine learning and statistical models used. The aforementioned geopolitical risk index worked on by modelling the stock price against the risk index using Vector Autoregressive Regression models. 

One of the simplest approaches to stock price modelling is to perform autoregressive modelling on purely the time series data for stock prices. In this approach, the only thing which is used to predict the prices are the past values of the stock. One common approach is to use an ARIMA model \cite{ariyo2014stock}, which is an autoregressive integrated moving average, which expresses Y as a linear combination of p past values multiplied by the weights. This aims to extract information from the previous p days to gain a balanced view of where the stock prices are likely to go on a day by day basis (or the time period being modelled).  

When predicting stock prices from dependant variables as opposed to just from the previous meetings, many different machine learning algorithms are used. Aside from classical regression algorithms, these include SVMs \cite{cao2003support}, Random Forests \cite{khaidem2016predicting}, and Neural Networks \cite{egeli2003stock}, along with more Bayesian approaches such as Naive Bayes \cite{khedr2017predicting}. These are generally regression predictors which aim to predict the stock prices themselves. However, there are algorithms which try and predict stock shift as opposed to prices, i.e. whether the market and prices have gone up or down \cite{nguyen2015sentiment}.  

More broadly, there is another concern with modelling on the stock market. When predicting any stock market, one of the major issues which arises is that in the long term the stock market always grows, for the S and P for example there is an annualised growth of on average 9.5\% (Macrotrends. S\&P 500 Historical Annual Returns) without the affect of inflation factored in. Thus any investor investing over a sufficiently long period of time will nearly always make money. The aim then, is to build models which would give better results than the normal stock market rise, and thus any strategy or predictions by any model would have to prove and provide better returns than just investing in the long term growth of the stock market.

A mixture of these algorithms were used alongside the GDELT algorithm to try and predict stock shift. 

